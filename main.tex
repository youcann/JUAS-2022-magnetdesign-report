\documentclass[10pt,a4paper,noendnumber=true]{scrartcl}
%german umlauts and localization
\usepackage[utf8]{inputenc} 
\usepackage[T1]{fontenc}
\usepackage[ngerman]{babel}
\usepackage[dvipsnames]{xcolor}
\usepackage{booktabs}
\usepackage{tabularx}

%figures etc.
\usepackage[pdftex]{graphicx}
\usepackage{standalone} %externalize files for faster compilation
\usepackage{float}

%mathematical symbols
\usepackage{latexsym} % special symbols 
\usepackage{amsmath,amssymb,amsthm}
\usepackage{textcomp} % supports the Text Companion fonts, 
\usepackage{mathrsfs}  % for script-like fonts in math mode
\usepackage{nicefrac} % nice fracs in text

%units
\usepackage{siunitx}
\DeclareSIUnit{\bar}{bar}

%tikz
\usepackage{tikz}
\usepackage{tikzscale}
\usepackage{pgfplots} 
\usepackage{pdflscape}
\usepackage[european]{circuitikz}
\pgfplotsset{compat=newest} 
\pgfplotsset{plot coordinates/math parser=false}
\usetikzlibrary{shapes.geometric}
\usetikzlibrary{arrows.meta}
\usetikzlibrary{arrows}
\usetikzlibrary{shapes.symbols,shadows}

%bib stuff
\usepackage[draft = false]{hyperref}
\usepackage{csquotes}
\usepackage[backend=biber,style=ieee]{biblatex}
%\addbibresource{bib.bib}

%align multi pgfplots
\pgfplotsset{yticklabel style={text width=3em,align=right}}

\usetikzlibrary{external}
\tikzexternalize[optimize=false,prefix=tikz/] % activate!

\usepackage{subfig}
\setlength\parindent{0pt}

\title{Magnet design for MedAustron}
\subtitle{}
\author{}
\date{\today}

\begin{document}
\maketitle

\section{Analytical}

\subsection{Magnet type decision}
Use H-type
\begin{itemize}
\item Mechanical rigid
\item Symmetrical
\item (-) hard to get the beam pipe in/out
\end{itemize}

\subsection{Aperture Height}
\begin{align}
    h &= 2\cdot h_\text{GFR} + 2\cdot d_\text{vacuum} + d_\text{tolerance} \\
      &= 2\cdot\SI{23}{\mm} + 2\cdot \SI{2}{\mm} + \SI{2}{\mm} = \SI{52}{\mm}
\end{align}

\subsection{Flux Density}
Total bending angle:
\begin{equation}
    \theta_\text{tot} = 3 \cdot \SI{36}{\degree} = \SI{108}{\degree}
\end{equation}

Total length
\begin{equation}
    l_\text{mag} = (l_\text{iron})_{max} + 2hk = \SI{0.340}{\meter} + 2\cdot 0.55 \cdot \SI{52}{\mm} = \SI{397.2}{\mm}
\end{equation}

\begin{equation}
    \theta_\text{mag} = \frac{l_\text{mag}}{\rho} \Rightarrow \rho = \frac{l_\text{mag}}{\theta_\text{mag}} = \frac{\SI{397.2}{\mm}}{\SI{0.6283}{\radian}} = \SI{0.642}{\meter}
\end{equation}

\begin{align}
    B_\text{min} &= \frac{(B \rho)_\text{min}}{\rho} = \frac{\SI{0.383}{\tesla\meter}}{\SI{0.642}{\meter}} = \SI{0.596}{\tesla}\\
    B_\text{max} &= \frac{(B \rho)_\text{max}}{\rho} = \frac{\SI{0.766}{\tesla\meter}}{\SI{0.642}{\meter}} = \SI{1.19}{\tesla}
\end{align}


\subsection{Pole width and yoke thickness}

\begin{equation}
    \frac{\Delta B}{B_0} = {1e-3}
\end{equation}

\begin{align}
    x_\text{unoptimized} &= 2\frac{a_\text{unoptimized}}{h} = -0.36 \ln\left[\frac{\Delta B}{B_0}\right] - 0.90 = 1.5898 \\
    x_\text{optimized} &= 2\frac{a_\text{optimized}}{h} = -0.14 \ln\left[\frac{\Delta B}{B_0}\right] -0.25 = 0.7171 \\
    a_\text{optimized} &= \frac{h\cdot x_\text{optimized}}{2} = \frac{\SI{52}{\mm} \cdot 0.7171}{2} = \SI{18.645}{\mm}
\end{align}

Pole width:
\begin{align}
    \text{GFRx'} = \text{GFRx} + \underbrace{\rho(1-\cos \nicefrac{\theta}{2}}_s = \SI{20}{\mm} + \SI{31.42}{\mm} 
    &= \SI{51.42}{\mm}
\end{align}

\begin{equation}
    w = 2\cdot \left(\text{GFRx'} + a_\text{optimized}\right) = \SI{140.13}{\mm}
\end{equation}

\subsection{Excitation Current}
\begin{align}
    (NI)_\text{dipole, min} &= \frac{B_\text{min}\,h}{2\,\mu_0} = \frac{\SI{0.596}{\tesla} \cdot \SI{52}{\mm}}{2\cdot \mu_0} = \SI{12.331}{\kilo\ampere}\\
    (NI)_\text{dipole, max} &= \frac{B_\text{max}\,h}{2\,\mu_0} = \frac{\SI{1.19}{\tesla} \cdot \SI{52}{\mm}}{2\cdot \mu_0} = \SI{24.621}{\kilo\ampere}
\end{align}

\subsection{Nominal Current and Turns}
Copper winding area
\begin{align}
A &= a^2- R^2 (4-\pi) - \pi \left(\frac{c}{2}\right)^2\\
A &= \SI{91.867}{\mm\squared}
\end{align}
with $a=\SI{11}{\mm}$, $R=\SI{1}{\mm}$, $c=\SI{6}{\mm}$.

Total allowed current: $I_\text{max,PS} = \SI{600}{\ampere}$

Maximum possible current limited by the winding cross section:
\begin{equation}
I_\text{max} = \SI{6}{\ampere\per\mm\squared} \cdot \SI{91.867}{\mm\squared} = \SI{551.202}{\ampere}
\end{equation}

Minimum number of turns:
\begin{align}
    N_{B\text{,min}} &= \left\lceil\frac{(NI)_\text{min}}{I_\text{max}}\right\rceil = 23\\
    N_{B\text{,max}} &= \left\lceil\frac{(NI)_\text{max}}{I_\text{max}}\right\rceil = 45
\end{align}

\subsection{Coil Parameters}
Coil width:
\begin{equation}
    w_\text{coil}=N_\text{horizontal} \cdot (a+d_\text{insulation}) + d_\text{epoxy} + d_\text{air} = \SI{124}{\mm}
\end{equation}

Coil height:
\begin{equation}
    h_\text{coil}=N_\text{vertical} \cdot (a+d_\text{insulation}) + d_\text{epoxy} + d_\text{air} = \SI{64}{\mm}
\end{equation}

with $N_\text{horizontal}=10$, $N_\text{vertical}=5$, $a=\SI{11}{\mm}$, $d_\text{insulation}=\SI{1}{\mm}$, $d_\text{epoxy}=\SI{2}{\mm}$ and $d_\text{air}=\SI{2}{\mm}$.


Average turn length:
\begin{equation}
    l_\text{avg}= \text{pole perimeter} + 4 \cdot w_\text{coil} + l_\text{mag} = \SI{32514.2}{\mm}
\end{equation}

\begin{equation}
pole perimeter = 2 \cdot (iron length + 2 \cdot pole width + coil width) = 9999.
\end{equation}



Resistance (with $\sigma=\sigma_\text{Cu}$):
\begin{equation}
    R_c=\frac{N\,l_\text{avg}}{A\,\sigma} = \frac{50\cdot \SI{999}{\mm}}{\SI{91.867}{\mm\squared}\cdot \SI{5.96e4}{\siemens\per\mm}}
\end{equation}

Voltage per magnet:
\begin{equation}
    V_m = m \cdot R_c\,I = 2 \cdot \SI{999}{\ohm} \cdot \SI{999}{\ampere} = \SI{999}{\volt} 
\end{equation}

Dissipated power per magnet:
\begin{equation}
    P_m = m \cdot R_c\,I^2 = \SI{999}{\watt}
\end{equation}

\subsection{Cooling}
Length of cooling circuit:
\begin{equation}
    l = \frac{K_c\,N\,l_\text{avg}}{K_w} = 
\end{equation}


With the maximum allowed pressure drop of $\Delta p = \SI{0.7}{\mega\pascal} = \SI{7}{\bar}$, the hydraulic diameter $d=\SI{6}{\mm}$ and using
\begin{equation}
\Delta p = 60\,l\frac{Q^{1.75}}{d^{4.75}},
\end{equation}
the maximum water flow $Q$ is
\begin{equation}
    Q = \left(\frac{\Delta p \cdot d^{4.75}}{60\cdot l}\right)^\frac{1}{1.75} = \SI{999}{\liter\per\minute}.
\end{equation}

Using this result and the maximum allowed temperature increase of $\Delta T=\SI{15}{\kelvin}$ together with
\begin{equation}
    Q = 14.3 \frac{P}{\Delta T} \cdot \num{1e-3},
\end{equation}
the maximum allowed dissipated power is
\begin{equation}
    P = \frac{Q \cdot \Delta T}{14.3 \cdot \num{1e-3}} = \SI{999}{\watt}.
\end{equation}

Average water velocity:
\begin{equation}
    u_\text{avg} = 16.67 \cdot \frac{Q}{A} = 16.67 \cdot \frac{4\cdot Q}{\pi d^2}
\end{equation}

Reynolds number (with $v=\SI{6.58e-7}{\meter\squared\per\second}$):
\begin{equation}
    R_e = d \cdot \frac{u_\text{avg}}{v} \cdot \num{1e-3} = \num{999}
\end{equation}
From $R_e><4000$ turbulent flow can/cannot be assumed.

\end{document}

