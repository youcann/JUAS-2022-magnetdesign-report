\documentclass[10pt,a4paper,noendnumber=true]{scrartcl}
%german umlauts and localization
\usepackage[utf8]{inputenc} 
\usepackage[T1]{fontenc}
\usepackage[english]{babel}
\usepackage[dvipsnames]{xcolor}
\usepackage{booktabs}
\usepackage{tabularx}

\usepackage{geometry}
\geometry{
	a4paper,
	total={170mm,257mm},
	left=20mm,
	top=20mm,
}

\newcommand\pro{\item[$+$]}
\newcommand\con{\item[$-$]}

%figures etc.
\usepackage[pdftex]{graphicx}
\usepackage{standalone} %externalize files for faster compilation
\usepackage{float}

%mathematical symbols
\usepackage{latexsym} % special symbols 
\usepackage{amsmath,amssymb,amsthm}
\usepackage{textcomp} % supports the Text Companion fonts, 
\usepackage{mathrsfs}  % for script-like fonts in math mode
\usepackage{nicefrac} % nice fracs in text

%units
\usepackage{siunitx}
\DeclareSIUnit{\bar}{bar}
\sisetup{output-product=\ensuremath{\cdot},exponent-product=\ensuremath{\cdot}}

%tikz
\usepackage{tikz}
\usepackage{tikzscale}
\usepackage{pgfplots} 
\usepackage{pdflscape}
\usepackage[european]{circuitikz}
\pgfplotsset{compat=newest} 
\pgfplotsset{plot coordinates/math parser=false}
\usetikzlibrary{shapes.geometric}
\usetikzlibrary{arrows.meta}
\usetikzlibrary{arrows}
\usetikzlibrary{math}
\usetikzlibrary{shapes.symbols,shadows}

%bib stuff
\usepackage[draft = false]{hyperref}
\usepackage{csquotes}
\usepackage[backend=biber,style=ieee]{biblatex}
%\addbibresource{bib.bib}

%align multi pgfplots
\pgfplotsset{yticklabel style={text width=3em,align=right}}

\usetikzlibrary{external}
\tikzexternalize[optimize=false,prefix=tikz/] % activate!

\usepackage{subfig}
\setlength\parindent{0pt}

\title{Magnet design for MedAustron}
\subtitle{}
\author{}
\date{\today}

\begin{document}
\maketitle

\section{Analytical}
\subsection{Magnet type decision}
The arguments for and against a H-type magnet are:
\begin{itemize}
\pro Mechanical rigid
\pro Symmetrical
\con Hard to get the beam pipe in and out
\end{itemize}

\subsection{Aperture Height}
The aperture height is given as the sum of the good field region $h_\text{GFR} = 2\cdot \text{GFR}_y$, the thickness of the vacuum pipe $d_\text{vacuum}$ and a tolerance for installation and thermal expansion $d_\text{tolerance}$ as
\begin{align}
    h &= h_\text{GFR} + 2\cdot d_\text{vacuum} + d_\text{tolerance} \\
      &= 2\cdot\SI{23}{\mm} + 2\cdot \SI{2}{\mm} + \SI{2}{\mm}\nonumber\\ 
      &= \SI{52}{\mm}. \nonumber
\end{align}

\subsection{Flux Density}
The total bending angle generated by all $m=3$ magnets is
\begin{equation}
    \theta_\text{tot} = 3 \cdot \theta_\text{mag} = 3 \cdot \SI{36}{\degree} = \SI{108}{\degree}.
\end{equation}

The magnetic length of the (dipole) magnet can be approximated by
\begin{align}
    l_\text{mag} &= l_\text{iron, max} + 2hk \\
    &= \SI{0.340}{\meter} + 2\cdot 0.55 \cdot \SI{52}{\mm}\nonumber\\
    &= \SI{397.2}{\mm}.\nonumber
\end{align}

With the definition of the radian ($\theta=\nicefrac{s}{\rho}$) and $s=l_\text{mag}$, the bending radius $\rho$ is

\begin{align}
    \theta_\text{mag} &= \frac{l_\text{mag}}{\rho} \\
    \Rightarrow \rho  &= \frac{l_\text{mag}}{\theta_\text{mag}}\\
    &= \frac{\SI{397.2}{\mm}}{\SI{0.6283}{\radian}} \nonumber\\
    &= \SI{0.642}{\meter}. \nonumber
\end{align}

With the given minimum and maximum $(B\rho)$ for a proton and a $C^{6+}$ beam, the minimum and maximum needed flux densities are
\begin{align}
    B_\text{min} &= \frac{(B \rho)_\text{min}}{\rho} = \frac{\SI{0.383}{\tesla\meter}}{\SI{0.642}{\meter}} = \SI{0.596}{\tesla}\\
    B_\text{max} &= \frac{(B \rho)_\text{max}}{\rho} = \frac{\SI{0.766}{\tesla\meter}}{\SI{0.642}{\meter}} = \SI{1.19}{\tesla}
\end{align}


\subsection{Pole width and yoke thickness}
With the given field quality inside the GFR $\nicefrac{\Delta B}{B_0} = \num{1e-3}$,
the optimized normalized pole overhang is\footnote{Compared to the unoptimized $x_\text{unoptimized}= 1.5898$}
\begin{align}
    x_\text{optimized} &= 2\frac{a_\text{optimized}}{h} \\
    &= -0.14 \ln\left[\frac{\Delta B}{B_0}\right] -0.25 \nonumber\\
    &= 0.7171. \nonumber
\end{align}

This equals a pole overhang of
\begin{align}
    a_\text{optimized} &= \frac{h\cdot x_\text{optimized}}{2} \\
    &= \frac{\SI{52}{\mm} \cdot 0.7171}{2} \nonumber\\
    &= \SI{18.645}{\mm}.\nonumber
\end{align}

Taking the sagitta $s$ into account and using
\begin{equation}
    \text{GFRx'} = \text{GFRx} + \underbrace{\rho(1-\cos \nicefrac{\theta}{2}}_s
    = \SI{20}{\mm} + \SI{31.42}{\mm} 
    = \SI{51.42}{\mm}, 
\end{equation}
the pole width $w$ computes to
\begin{align}
	w & = 2\cdot \left(\text{GFRx'} + a_\text{optimized}\right)            \\
	  & = 2\cdot \left(\SI{51.42}{\mm} + \SI{18.645}{\mm}\right) \nonumber \\
	  & = \SI{140.13}{\mm}. \nonumber
\end{align}


\subsection{Excitation Current}
The ampereturns are computed as
\begin{align}
    (NI)_\text{dipole, min} &= \frac{B_\text{min}\,h}{2\,\mu_0} = \frac{\SI{0.596}{\tesla} \cdot \SI{52}{\mm}}{2\cdot \mu_0} = \SI{12.331}{\kilo\ampere}\\
    (NI)_\text{dipole, max} &= \frac{B_\text{max}\,h}{2\,\mu_0} = \frac{\SI{1.19}{\tesla} \cdot \SI{52}{\mm}}{2\cdot \mu_0} = \SI{24.621}{\kilo\ampere}
\end{align}

\begin{figure}[H]
\centering
\includegraphics[width=0.5\textwidth]{img/winding.tikz}
\caption{Cross section of the copper winding showing  insulation(\textcolor{orange}{$\blacksquare$}) and water channel(\textcolor{blue!20}{$\blacksquare$})}
\label{fig:winding}
\end{figure}

With the given dimensions of the copper winding (see \autoref{fig:winding}) and using 
\begin{align}
A_\text{rounded edges} &= 4 \cdot  A_\text{rounded edge} = 4 \frac{r_o^2 (4-\pi)}{4} = r_o^2(4-\pi) \\
A_\text{water channel} &= \pi r_c^2
\end{align}

the copper cross section area is
\begin{align}
	A_\text{copper} & = a^2 - \underbrace{r_o^2(4-\pi)}_{A_\text{rounded edges}} - \underbrace{\pi r_c^2}_{A_\text{water channel}} \\
	  & = \SI{121}{\mm\squared} - \SI{0.8584}{\mm\squared} - \SI{28.274}{\mm\squared} \nonumber                      \\
	  & = \SI{91.867}{\mm\squared} \nonumber
\end{align}

With the given maximum current density of $j_\text{max}=\SI{6}{\ampere\per\mm\squared}$, the maximum possible current in the winding is
\begin{align}
I_\text{max} &= j_\text{max} \cdot A_\text{copper} \\
 &=\SI{6}{\ampere\per\mm\squared} \cdot \SI{91.867}{\mm\squared} \nonumber\\ 
 &= \SI{551.202}{\ampere}. \nonumber
\end{align}

With the calculated ampereturns, the necessary minimum number of turns are then
\begin{align}
    N_{B_\text{min}} &= \left\lceil\frac{(NI)_\text{min}}{I_\text{max}}\right\rceil = 23\\
    N_{B_\text{max}} &= \left\lceil\frac{(NI)_\text{max}}{I_\text{max}}\right\rceil = 45
\end{align}

As it is easier to lower the current then to change the number of turns, the minimum number of turns should be $N_{B_\text{max}}$.

To full fill the empirical good practice $\nicefrac{N_\text{horizontal}}{N_\text{vertical}}=2$, the number of turn is set to
\begin{equation}
    N=N_\text{horizontal} \times N_\text{vertical} = 10 \times 5 = 50.
\end{equation}

With the new $N=50$, the winding current is
\begin{equation}
    I = \frac{(NI)_\text{max}}{50} = \SI{492.42}{\ampere},
\end{equation}
which is well below the maximum current of the power supply ($I_\text{max}=\SI{600}{\ampere}$).

\subsection{Coil Parameters}
The coil width and height are given by (see \autoref{fig:coil})
\begin{equation}
    w_\text{coil}=N_\text{horizontal} \cdot (a+d_\text{insulation}) + 2\cdot(d_\text{epoxy} + d_\text{air}) = \SI{128}{\mm}
\end{equation}
and
\begin{equation}
    h_\text{coil}=N_\text{vertical} \cdot (a+d_\text{insulation}) + 2\cdot(d_\text{epoxy} + d_\text{air}) = \SI{68}{\mm}
\end{equation}
using $N_\text{horizontal}=10$, $N_\text{vertical}=5$, $a=\SI{11}{\mm}$, $d_\text{insulation}=\SI{1}{\mm}$, $d_\text{epoxy}=\SI{2}{\mm}$ and $d_\text{air}=\SI{2}{\mm}$.

\begin{figure}[H]
\centering
\includegraphics[width=\textwidth]{img/coil.tikz}
\caption{Cross section of one coil showing copper(\textcolor{black}{$\square$}), insulation(\textcolor{orange}{$\blacksquare$}), 
the water channels(\textcolor{blue!30}{$\blacksquare$}),
the coil epoxy coating(\textcolor{red!30}{$\blacksquare$}) and
the air gap around the coil(\textcolor{green!30}{$\blacksquare$})}
\label{fig:coil}
\end{figure}

With the pole perimeter
\begin{equation}
	p = 2 \cdot l\text{iron} + 2 \cdot w = \SI{960}{\mm},
\end{equation}
the average turn length is ($l\text{iron} = \SI{340}{\mm}$, $w = \SI{140.13}{\mm}$ and $w_\text{coil} = \SI{128}{\mm}$)
\begin{equation}
    l_\text{avg}= p + 4 \cdot w_\text{coil} = \SI{1472}{\mm}.
\end{equation}

The resistance of one coil winding is (with $1/\sigma_\text{Cu}=\SI{1.72}{\micro\ohm\cm}$)
\begin{align}
R_c&=\frac{N\cdot l_\text{avg}}{A\cdot\sigma_\text{Cu}} \\
&= \frac{50 \cdot \SI{1472}{\mm}}{\SI{91.867}{\mm\squared}\cdot \nicefrac{1}{\SI{1.72}{\micro\ohm\cm}}} \nonumber\\
&= \SI{13.442}{\milli\ohm} \nonumber
\end{align}

With the number of coils per magnet $m$, the DC steady-state voltage per magnet is
\begin{align}
V_m &= I \cdot m\,R_c \\
&= \SI{492.42}{\ampere} \cdot 2 \cdot \SI{0.016}{\ohm} \nonumber\\
&= \SI{13.197}{\volt}.\nonumber
\end{align}

The total voltage over all magnets is (number of magnets $M$)
\begin{align}
V_\text{total} &= M \cdot V_m \\
&= 3 \cdot \SI{16.197}{\volt} \nonumber\\
&= \SI{39.591}{\volt}.\nonumber
\end{align}
This is well below the maximum output voltage of the power supply ($V_\text{total, max}=\SI{80}{\volt}$).

The dissipated power in one magnet is calculated to
\begin{align}
P_m &= V_m \cdot I = I^2 \cdot m\,R_c \\
&= \SI{13.197}{\volt} \cdot \SI{492.42}{\ampere} \nonumber\\
&= \SI{6.498}{\kilo\watt}.\nonumber
\end{align}

\subsection{Cooling}
With the given $\Delta T=\SI{15}{\kelvin}$ and the dissipated power $P_m=\SI{6.498}{\kilo\watt}$, the total required water flow in one magnet is
\begin{equation}
    Q = 14.3 \frac{P_m}{\Delta T} \cdot \num{1e-3} = \SI{6.2}{\liter\per\minute}
\end{equation}

Using the length of cooling circuit for one coil is ($K_c=1$ and $K_w=1$)
\begin{align}
    l &= \frac{K_c\,N\,l_\text{avg}}{K_w} \\
    &= \frac{1\cdot50\cdot\SI{1472}{\mm}}{1} \nonumber\\ 
    &= \SI{73.6}{\meter}. \nonumber
\end{align}

The pressure drop over a cooling circuit is given by ($d=2r_c=\SI{6}{\mm}$)
\begin{equation}
    \Delta p = 60 \cdot l \cdot \frac{Q^{1.75}}{d^{4.75}} 
\end{equation}

This can be interpreted a (non-linear) hydraulic equivalent to Ohm's law in the electrical domain:
\begin{equation}
	\Delta p = \underbrace{\frac{60 \cdot l}{d^{4.75}}}_R \cdot Q^{1.75} = R \cdot Q^{1.75}
\end{equation}

The hydraulic resistance of one coil is given (in AU) as
\begin{equation}
R= \frac{60 \cdot l}{d^{4.75}} = 0.888
\end{equation}

To cool one magnet, the cooling circuits can either be connected in series or in parallel (see \autoref{fig:cool}).

\begin{figure}[H]
	\centering\setcounter{subfigure}{0}
	\subfloat[Series connection]{\includegraphics[width=0.8\textwidth]{img/coolingSeries.tikz}}\\
	\subfloat[Parallel connection]{\includegraphics[width=0.8\textwidth]{img/coolingParallel.tikz}}
	\caption{Possible cooling circuit configurations}\label{fig:cool}
\end{figure}

For both cases, the pressure drops are
\begin{align}
\Delta p_\text{Series} &= 2 \cdot 0.888 \cdot 6.2^{1.75} = \SI{43.26}{\bar} \\
\Delta p_\text{Parallel} &= 0.888 \cdot 3.1^{1.75} = \SI{6.43}{\bar}
\end{align}

As the series connection pressure drop exceeds the limit of the pump ($\Delta p_\text{max}=\SI{7}{\bar}$), the parallel connection is used.

The average flow velocity in this case is
\begin{equation}
    u_\text{avg} = 16.67 \cdot \frac{Q}{A} = 16.67 \cdot \frac{4\cdot Q}{\pi d^2} = 16.67 \cdot \frac{4 \cdot \SI{3.1}{\liter\per\minute}}{\pi \cdot (\SI{6}{\mm})^{2}} = \SI{1.827}{\meter\per\second}.
\end{equation}

Reynolds number (with $v=\SI{6.58e-7}{\meter\squared\per\second}$) is computed as
\begin{equation}
    R_e = d \cdot \frac{u_\text{avg}}{v} \cdot \num{1e-3} = \num{16660}.
\end{equation}
From $R_e>4000$ turbulent flow can be assumed for the flow in the water channels, which is needed for equal heat distribution.



\newpage
\section{Numerical}

\begin{figure}[H]
\centering
\includegraphics[width=\textwidth]{img/quaterH.tikz}
\caption{Input data for numerical simulation; all measurements in \si{\mm}; drawing not to scale}
\end{figure}

\begin{table}[H]
\centering
\caption{Relevant magnet parameters}
\begin{tabular}{ll}
\toprule
Name & Value \\
\midrule
Flux density $B$ & \SIrange{0.596}{1.19}{\tesla}\\
Gap height $h$ & \SI{52}{\mm}\\
Pole width $w$ & \SI{140}{\mm}\\
Ampereturns $NI$ & \SIrange{12.3}{24.6}{\kilo\ampere}\\
\bottomrule
\end{tabular}
\end{table}



\end{document}

