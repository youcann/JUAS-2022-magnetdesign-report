\documentclass[10pt,a4paper,noendnumber=true]{scrartcl}
%german umlauts and localization
\usepackage[utf8]{inputenc} 
\usepackage[T1]{fontenc}
\usepackage[english]{babel}
\usepackage[dvipsnames]{xcolor}
\usepackage{booktabs}
\usepackage{tabularx}

\usepackage{geometry}
\geometry{
	a4paper,
	total={170mm,257mm},
	left=20mm,
	top=20mm,
}

%figures etc.
\usepackage[pdftex]{graphicx}
\usepackage{standalone} %externalize files for faster compilation
\usepackage{float}

%mathematical symbols
\usepackage{latexsym} % special symbols 
\usepackage{amsmath,amssymb,amsthm}
\usepackage{textcomp} % supports the Text Companion fonts, 
\usepackage{mathrsfs}  % for script-like fonts in math mode
\usepackage{nicefrac} % nice fracs in text

%units
\usepackage{siunitx}
\DeclareSIUnit{\bar}{bar}
\sisetup{output-product=\ensuremath{\cdot},exponent-product=\ensuremath{\cdot}}

%tikz
\usepackage{tikz}
\usepackage{tikzscale}
\usepackage{pgfplots} 
\usepackage{pdflscape}
\usepackage[european]{circuitikz}
\pgfplotsset{compat=newest} 
\pgfplotsset{plot coordinates/math parser=false}
\usetikzlibrary{shapes.geometric}
\usetikzlibrary{arrows.meta}
\usetikzlibrary{arrows}
\usetikzlibrary{math}
\usetikzlibrary{shapes.symbols,shadows}

%bib stuff
\usepackage[draft = false]{hyperref}
\usepackage{csquotes}
\usepackage[backend=biber,style=ieee]{biblatex}
%\addbibresource{bib.bib}

%align multi pgfplots
\pgfplotsset{yticklabel style={text width=3em,align=right}}

\usetikzlibrary{external}
\tikzexternalize[optimize=false,prefix=tikz/] % activate!

\usepackage{subfig}
\setlength\parindent{0pt}

\title{Magnet design for MedAustron}
\subtitle{}
\author{}
\date{\today}

\begin{document}
\maketitle

\section{Analytical}

\subsection{Magnet type decision}
Use H-type
\begin{itemize}
\item Mechanical rigid
\item Symmetrical
\item (-) hard to get the beam pipe in/out
\end{itemize}

\subsection{Aperture Height}
\begin{align}
    h &= 2\cdot h_\text{GFR} + 2\cdot d_\text{vacuum} + d_\text{tolerance} \\
      &= 2\cdot\SI{23}{\mm} + 2\cdot \SI{2}{\mm} + \SI{2}{\mm} = \SI{52}{\mm}
\end{align}

\subsection{Flux Density}
Total bending angle:
\begin{equation}
    \theta_\text{tot} = 3 \cdot \SI{36}{\degree} = \SI{108}{\degree}
\end{equation}

Total length
\begin{equation}
    l_\text{mag} = (l_\text{iron})_{max} + 2hk = \SI{0.340}{\meter} + 2\cdot 0.55 \cdot \SI{52}{\mm} = \SI{397.2}{\mm}
\end{equation}

\begin{equation}
    \theta_\text{mag} = \frac{l_\text{mag}}{\rho} \Rightarrow \rho = \frac{l_\text{mag}}{\theta_\text{mag}} = \frac{\SI{397.2}{\mm}}{\SI{0.6283}{\radian}} = \SI{0.642}{\meter}
\end{equation}

\begin{align}
    B_\text{min} &= \frac{(B \rho)_\text{min}}{\rho} = \frac{\SI{0.383}{\tesla\meter}}{\SI{0.642}{\meter}} = \SI{0.596}{\tesla}\\
    B_\text{max} &= \frac{(B \rho)_\text{max}}{\rho} = \frac{\SI{0.766}{\tesla\meter}}{\SI{0.642}{\meter}} = \SI{1.19}{\tesla}
\end{align}


\subsection{Pole width and yoke thickness}

\begin{equation}
    \frac{\Delta B}{B_0} = {1e-3}
\end{equation}

\begin{align}
    x_\text{unoptimized} &= 2\frac{a_\text{unoptimized}}{h} = -0.36 \ln\left[\frac{\Delta B}{B_0}\right] - 0.90 = 1.5898 \\
    x_\text{optimized} &= 2\frac{a_\text{optimized}}{h} = -0.14 \ln\left[\frac{\Delta B}{B_0}\right] -0.25 = 0.7171 \\
    a_\text{optimized} &= \frac{h\cdot x_\text{optimized}}{2} = \frac{\SI{52}{\mm} \cdot 0.7171}{2} = \SI{18.645}{\mm}
\end{align}

Pole width:
\begin{align}
    \text{GFRx'} = \text{GFRx} + \underbrace{\rho(1-\cos \nicefrac{\theta}{2}}_s = \SI{20}{\mm} + \SI{31.42}{\mm} 
    &= \SI{51.42}{\mm}
\end{align}

\begin{equation}
    w = 2\cdot \left(\text{GFRx'} + a_\text{optimized}\right) = \SI{140.13}{\mm}
\end{equation}

\subsection{Excitation Current}
\begin{align}
    (NI)_\text{dipole, min} &= \frac{B_\text{min}\,h}{2\,\mu_0} = \frac{\SI{0.596}{\tesla} \cdot \SI{52}{\mm}}{2\cdot \mu_0} = \SI{12.331}{\kilo\ampere}\\
    (NI)_\text{dipole, max} &= \frac{B_\text{max}\,h}{2\,\mu_0} = \frac{\SI{1.19}{\tesla} \cdot \SI{52}{\mm}}{2\cdot \mu_0} = \SI{24.621}{\kilo\ampere}
\end{align}

\subsection{Nominal Current and Turns}

\begin{figure}[H]
\centering
\includegraphics[width=\textwidth]{img/winding.tikz}
\caption{Cross section}
\end{figure}


Copper winding area
\begin{align}
A &= a^2- R^2 (4+\pi) - \pi \left(\frac{c}{2}\right)^2\\
A &= \SI{91.867}{\mm\squared}
\end{align}
with $a=\SI{11}{\mm}$, $R=\SI{1}{\mm}$, $c=\SI{6}{\mm}$.

Total allowed current: $I_\text{max,PS} = \SI{600}{\ampere}$

Maximum possible current limited by the winding cross section:
\begin{equation}
I_\text{max} = \SI{6}{\ampere\per\mm\squared} \cdot \SI{91.867}{\mm\squared} = \SI{551.202}{\ampere}
\end{equation}

Minimum number of turns:
\begin{align}
    N_{B\text{,min}} &= \left\lceil\frac{(NI)_\text{min}}{I_\text{max}}\right\rceil = 23\\
    N_{B\text{,max}} &= \left\lceil\frac{(NI)_\text{max}}{I_\text{max}}\right\rceil = 45
\end{align}

To full fill the empirical good practice $\nicefrac{N_\text{horizontal}}{N_\text{vertical}}=2$,
\begin{equation}
    N=N_\text{horizontal} \times N_\text{vertical} = 10 \times 5 = 50
\end{equation}
is chosen.

New current:
\begin{equation}
    \frac{(NI)_\text{max}}{I} = N \Rightarrow I = \frac{(NI)_\text{max}}{50} = \SI{492.42}{\ampere}
\end{equation}

\subsection{Coil Parameters}
Coil width:
\begin{equation}
    w_\text{coil}=N_\text{horizontal} \cdot (a+d_\text{insulation}) + 2\cdot(d_\text{epoxy} + d_\text{air}) = \SI{128}{\mm}
\end{equation}

Coil height:
\begin{equation}
    h_\text{coil}=N_\text{vertical} \cdot (a+d_\text{insulation}) + 2\cdot(d_\text{epoxy} + d_\text{air}) = \SI{68}{\mm}
\end{equation}

with $N_\text{horizontal}=10$, $N_\text{vertical}=5$, $a=\SI{11}{\mm}$, $d_\text{insulation}=\SI{1}{\mm}$, $d_\text{epoxy}=\SI{2}{\mm}$ and $d_\text{air}=\SI{2}{\mm}$.


Average turn length:
\begin{equation}
    l_\text{avg}= \text{pole perimeter} + 4 \cdot w_\text{coil} = \SI{1472}{\mm}
\end{equation}

\begin{equation}
pole perimeter = 2 \cdot iron length + 2 \cdot pole width = \SI{960}{\mm}
\end{equation}

with $iron length = \SI{340}{\mm}$, $pole width = \SI{140.13}{\mm}$, $w_\text{coil} = \SI{128}{\mm}$

\begin{figure}[H]
\centering
\includegraphics[width=\textwidth]{img/coil.tikz}
\caption{Coil cross section}
\end{figure}

Total resistance of the winding (wit $1/\sigma_\text{Cu}=\SI{1.72}{\micro\ohm\cm}$)
\begin{equation}
R_c=\frac{N\cdot l_\text{avg}}{A\cdot\sigma_\text{Cu}} = \frac{50 \cdot \SI{1472}{\mm}}{\SI{91.867}{\mm\squared}\cdot \nicefrac{1}{\SI{1.72}{\micro\ohm\cm}}} = \SI{13.442}{\milli\ohm}
\end{equation}

Voltage per magnet:
\begin{equation}
V_m = I \cdot m\,R_c = \SI{492.42}{\ampere} \cdot 2 \cdot \SI{0.016}{\ohm} = \SI{13.197}{\volt}
\end{equation}

Total voltage (number of magnets $M$):
\begin{equation}
V_\text{total} = M \cdot V_m = 3 \cdot \SI{16.197}{\volt} = \SI{39.591}{\volt}
\end{equation}

Dissipated power:
\begin{equation}
P_m = V_m \cdot I = I^2 \cdot m\,R_c = \SI{13.197}{\volt} \cdot \SI{492.42}{\ampere} = \SI{6.498}{\kilo\watt}
\end{equation}

\subsection{Cooling}
With $\Delta T=\SI{15}{\kelvin}$ and the dissipated power $P_m=\SI{6.498}{\kilo\watt}$, the required water flow is
\begin{equation}
    \frac{Q}{2} = 14.3 \frac{P_m}{\Delta T} \cdot \num{1e-3} = \SI{3.098}{\liter\per\minute}
\end{equation}
%inside si you can just use "e" for the exponent :)




With the water channel diameter $d=2r_c=\SI{6}{\mm}$, the average flow velocity is
\begin{equation}
    u_\text{avg} = 16.67 \cdot \frac{Q}{A} = 16.67 \cdot \frac{4\cdot Q}{\pi d^2} = 16.67 \cdot \frac{4 \cdot \SI{3.098}{\liter\per\minute}}{\pi \cdot (\SI{6}{\mm})^{2}} = \SI{1.827}{\meter\per\second}.
\end{equation}

Using the length of cooling circuit ($K_c=2$ and $K_w=1$)
\begin{equation}
    l = \frac{K_c\,N\,l_\text{avg}}{K_w} = \frac{1\cdot50\cdot\SI{1472}{\mm}}{1} = \SI{73.6}{\meter},
\end{equation}
the pressure drop is
\begin{equation}
    \Delta p 
    = 60 \cdot l \cdot \frac{Q^{1.75}}{d^{4.75}} 
    = 60 \cdot \SI{73.6}{\meter} \cdot \frac{(3.098)^{1.75}}{(6)^{4.75}}
    = \SI{6.43}{\bar}
\end{equation}

Reynolds number (with $v=\SI{6.58e-7}{\meter\squared\per\second}$):
\begin{equation}
    R_e = d \cdot \frac{u_\text{avg}}{v} \cdot \num{1e-3} = \num{16660}
\end{equation}
From $R_e>4000$ turbulent flow can be assumed.





\newpage
\section{Numerical}

\begin{figure}[H]
\centering
\includegraphics[width=\textwidth]{img/quaterH.tikz}
\caption{Input data for numerical simulation; all measurements in \si{\mm}; drawing not to scale}
\end{figure}

\begin{table}[H]
\centering
\caption{Relevant magnet parameters}
\begin{tabular}{ll}
\toprule
Name & Value \\
\midrule
Flux density $B$ & \SI{999}{\tesla}\\
Gap height $h$ & \SI{999}{\mm}\\
Pole width $w$ & \SI{999}{\mm}\\
Ampereturns $NI$ & \SI{999}{\ampere}\\
\bottomrule
\end{tabular}
\end{table}



\end{document}

